\chapter{Empiri}
\label{Ch:3}

Gennem dette kapitel belyses opgavens empiriske design, herunder de valg der er truffet vedrørende den empiri som danner grundlaget for undersøgelsen i specialet. Ligeledes vil kapitlet reflekterer over nogle af de udfordringer som kunne påpeges ved det valgte empiriske design. Kapitlet er struktureret således at afsnit \vref{sec:3.0} beskriver de forskellige muligheder for indsamling af empiri til specialet, afsnit \vref{sec:3.1} beskriver det valgte empiriske forskingsdesign med de fordele det afstedkommer. Afsnit \vref{sec:3.2} beskriver fasen med indsamling af det empiriske grundlag, og sluttelig forholder afsnit \vref{sec:3.3} sig til de begrænsninger der kan være forbundet med det empiriske forskningsdesign.

\section{Empiriske muligheder}
\label{sec:3.0}
Fundamentalt set er det muligt at forestille sig to typer af empiri som grundlag for denne opgave. Den ene udelukker ikke nødvendigvis den anden. Først og fremmst kunne man forestille sig at man interviewede en række interessenter inden for feltet, både elever og lærere med fokus på motivation for faget fysik og betydningen for den oplevede faglige udvikling af de skriftlige kompetencer i faget. På baggrund af disse interview kunne man undersøge og den oplevede motivation ved praktisk arbejde faktisk går hånd i hånd med de skriftlige kompetencer. Dette kunne eventuelt kobles med en række observationer af undervisning. Dette vil give opgaven et kavlitativt fundament med nogle elev og nogle lærere udsagn til at belyse problemstillingen.


\section{Empirisk forskningsdesign}
\label{sec:3.1}

\section{Indsamling af empiriske data}
\label{sec:3.2}

\section{Empiriske begrænsninger}
\label{sec:3.3}


\section{Empiri grundlag}
%\label{sec:3.1}
I forbindelse med denne opgave er der indsamlet empiri i en STX klasse på Viborg Katedralskole. Klassen som undersøges i dette projekt er en stærkt naturvidenskabelig klasse med Fysik på A-niveau. Forud for det forløb som empirien er indsamlet i er eleverne blevet adspurgt om deres faglige motivation. Med udgangspunkt i følgende tre selvevaluerings spørgsmål:
\begin{enumerate}
	\item Hvad er mit forhold til de naturvidendsabelige fag?
	\item Hvad er min personlige motivation for de naturvidenskabelige fag?
	\item Hvordan motiveres jeg til at yde den optimale indsats i timerne?
\end{enumerate}
Af de tre spørgsmål er det spørgsmål to som er det særligt interessante, hvor eleverne skriftligt har skullet uddybe deres holdninger. Til besvarelsen har vi anvendt Lectios Elev feedback som giver et fortroligt rum hvor kun læren og den enktelte elev kan se hvad der skrives. 

\section{Indsamling af empiri}
%\label{sec:3.2}
I forbindelse med arbejdet med at indsamle empiri falder det i to dele. Dene ene del vil være i form at undersøgelse af tegn på forbedring af den skriftlige kompetence for eleverne i klassen. Dette gøres simpelthen ved at se på elevernes skriftlige produkter, og følge deres udvikling over tid. Dette sammen holdes eventuelt med lignende skriftlige produkter fra elever som ikke har været undervist efter IBSE tanken og med fokus på brugen af SWH som styrring af det skriftlige arbejde.

Den anden del består i at undersøge eleverne motivation for det praktiske arbejde i faget fysik. Her beder vi eleverne om at udfylde et simpelt spørgeskema hver gang de har været i laboratoriet , for at få en ide om hvorvidt elevernes motivation øges ved praktisk arbejde i faget eller eleverne finder arbejdet med IBSE så svært at de opnår den modsatrettede effekt.

\subsection{Spørgeskema om motivation}
\label{sub:3.2.a}
I forbindelse med elevernes selvevaluering af deres motivations niveau, er jeg kommet frem til at eleverne skal vurdere følgende fem udsagn på følgende skala: Meget, lidt, hverken eller, mindre eller slet ikke. De fem udsagn er følgende:
\begin{itemize}
	\item Jeg har let ved at gennemskue hvad jeg skal i laboratoriet.
 	\item Jeg har et øget fagligt udbytte af de åbne problemstillinger.
	\item Jeg har en bedre forståelse af den teori der arbejdes med som følge af laboratorie arbejdet.
	\item Jeg føler at skriveprocessen er nemmere når jeg selv har designet forsøget.
	\item Jeg føler at det praktiske arbejde i laboratoriet, øger min faglige motivation.
\end{itemize}
På baggrund af elevernes svar i spørgeskemaet kan der blive tale om at udtage enkelte elever til et interview for at få uddybende svar vedrørende deres tanker om motivation for faget og deres egen udvikling af skriftlig kompetence i fysik faget i særdeleshed og naturfag som helhed.

\section{Liste over empiriske muligheder}
%\label{sec:3.3}

Der er mange muligheder når vi skal se på udviklingen af elevernes faglige motivation samt udviklingen af elevernes skriftlige kompetence. Herunder følger en liste som den ser ud pr. d.d. og denne er krydret med indspark fra \citet{Brinkmann2015}.
\begin{enumerate}
	 \item {\bfseries \color{red}Test} - jeg har tænkt på at man kunne gennemfører nogle tests som tester elevernes viden om faglige problemstillinger før under og efter et forløb med fokus på skriftlighed og UBNU. En af udfordringerne her er at kunne sammenligne for at se på om de forbedre sig mere end andre, og hvordan måler vi egentlig skriftlig kompetence og motivation for faget?
	 \item {\bfseries \color{red}Aflevering} - Jeg har tænkt på om man kunne se på elevernes afleveringer sammenlignet med lignende klasser som ikke har et særligt fokus på at kører deres forløb gennem UBNU og med særligt fokus på skriftlighed og formativ feedback. Igen så er her nogle faldgrupper i form af at vi kan komme til at sammenligne æbler og pærer og ikke nødvendigvis være istand til at påvise det vi tror vi ser. 
	 \item {\bfseries \color{red} Spørgeskema} - En tredje model er at bede eleverne udfylde et super kort spørgeskema fra tid til anden gennem forløbene for at se på hvordan elevernes faglige motivation er koblet til det at lave praktisk arbejde i faget. Her vil der i  mindre grad være behov for at sammenligne på tværs af klasser selvom det vil give en bedre baseline. Men der er igen mange faktorer som kan spille ind på elevernes oplevede motivation for et fag. Herunder både ydre og indre motivation, jf \citet{Buhl2010}. Fordelen ved at have et spørgeskema er at man får samlet et statisktisk materiale som er forholdsvist til at have med at gøre.
	\item {\bfseries \color{red} Interview} - En anden mulighed kunne være at interviewe nogle elever om deres motivation for faget fysik og så efterfølgende se på hvordan disse elevers oplevede motivation stemmer overens med lærernes oplevelse af elevernes motivation i forhold til faget og se på om motivation, praktisk arbejde og skriftlighed faktisk går hånd-i-hånd.
	\item {\bfseries \color{red} Selvevaluering} - En alternativ metode vi har snakket om var at lade eleverne selv evaluerere deres motivation for at gennemfører det konkrete skriftlige produkt således at man i højere grad fik eleverne egne holdninger til den måde at arbejde på i spil. Dette kunne også gøre på tværs af hold på en årgang og så på baggrund af elevernes svar udtage personer til et uddybende interview.
	\item {\bfseries \color{red} Observationer} - Der tænkes at skulle foretages observationer i de klasser hvor jeg kommer til at hente forsøgspersoner til projektet således at jeg kan se hvordan de bliver undervist i forhold til deres praktiske arbejde - altså om de gennemfører øvelser efter UBNU tanken mv.
\end{enumerate}