\chapter{Metode, design \& data}
\label{Ch:3}

%Gennem dette kapitel belyses opgavens empiriske design, herunder de valg der er truffet vedrørende den empiri som danner grundlaget for undersøgelsen i specialet. Ligeledes vil kapitlet reflekterer over nogle af de udfordringer som kunne påpeges ved det valgte empiriske design. Kapitlet er struktureret således at afsnit \vref{sec:3.0} beskriver de forskellige muligheder for indsamling af empiri til specialet, afsnit \vref{sec:3.1} beskriver det valgte empiriske forskingsdesign med de fordele det afstedkommer. Afsnit \vref{sec:3.2} beskriver fasen med indsamling af det empiriske grundlag, og sluttelig forholder afsnit \vref{sec:3.3} sig til de begrænsninger der kan være forbundet med det empiriske forskningsdesign.

\section{Empirisk metode}
\label{sec:3.0}
Til belysning af problemstillingen anvendes aktionsforskning. Med aktionsforskning er det muligt at følge en meget lille poppulation af elever på tæt hold, hvilket er en fordel i denne kontekt, Herved er det muligt at brede undersøgelsen ud på et senere tidspunkt hvis det ser ud til at der er grundlag for dette. En gruppe elever er blevet udvalgt til at danne grundlaget for undersøgelsen af hvordan det påvirker deres skriftlige kompetencer at man anvender IBSE og SWH i undervisningen. 
Den udvalgte klasse er herefter blevet undervist med et særligt fokus på IBSE, og med en klar kobling til SWH. Fra klassen er der blevet udvalgt skriftligt arbejde fra to grupper, som efterfølgende er blevet analyseret, gennem flere afleveringer. På denne måde kan vi se på udviklingen af elevernes skriftlige kompetencer. For at undersøge om det er specielt for netop denne klasse er afleveringer fra to andre klasser ligeledes blevet analyseret fra to tilfældigt valgte grupper. 
Ydermere indsamles der data fra et spørgskema som forholder sig til om eleverne føler sig klædt på til arbejdet med det skriftlige produkt.


\section{Empirisk design}
\label{sec:3.1}

Den forskningsmetode som anvendes i denne opgave vil naturligt være aktionsforskning. Grundlaget for dette valg er af praktiske hensyn, på den måde er det muligt at afprøve forskellige tiltag på et mindre ensemble for derefter at undersøge effekterne af de ændringer der er foretaget blandt målgruppen. Herefter er det muligt at korrigere og foretage nye observationer af effekterne af den afprøvede ændring. En anden begrænsende faktor her har været tid. Skulle man have foretaget undersøgelserne på et større ensemble så ville det have krævet langt flere undervisere som var indvolveret i forskningen. Dette har ikke været muligt, hvorfor valget er faldet på at studerer effekterne i en enkelt klasse med 30 elever.


Som basis for undersøgelsen af elevernes skriftlige produkter er to grupper af elever blevet udvalgt, således at det er muligt at se på deres skriftlige udvikling over tid. 

\section{Empiriske data}
\label{sec:3.2}

Indsamlingen af empiri kommer til at falde i flere dele. Den første del bliver at indsamle svar på et kort spørgsskema om hvorvidt eleverne føler sig klædt på til den skriftlige opgave som de står overfor i forhold til det eksperiment de netop har udført. Spørgeskemaet udfyldes af eleverne når de har lavet deres eksperimenter og er klar til at påbegynde skriveprocessen. Det er uafhængigt af om eleverne skal skrive en journal eller en rapport om de eksperiment de netop har gennemført. 
Spørgeskemaet er blevet gennemført i alle 1.g klasser på Viborg Katedralskole hvilket betyder at der er 349 mulige respondendter til spørgskemaet, af dem har 117 afgivet en besvarelse hvilket betyder at spørgeskemaet har en svarprocent på 33,5\%. 

\subsection{Spørgeskema om motivation for skriftligt arbejde}
\label{sub:3.2.a}
Eleverens selvevaluering af deres eget motivations niveau, har mundet ud i en udformning af et spørgeskema med fem udsagn som eleverne skal vurdere på en syv trinsskala. Hvor syv er meget enig og et er meget uenig. De fem udsagn som eleverne skal vurdere er følgende:
\begin{enumerate}
	\item Jeg har let ved at gennemskue hvad jeg skal i laboratoriet.
 	\item Jeg har et øget fagligt udbytte af de åbne problemstillinger.
	\item Jeg har en bedre forståelse af den teori der arbejdes med som følge af laboratorie arbejdet.
	\item Jeg føler at skriveprocessen er nemmere når jeg selv har designet forsøget.
	\item Jeg føler at det praktiske arbejde i laboratoriet, øger min faglige motivation.
\end{enumerate}
Spørgeskemaet gennemføres i en eksperimentel lektion umiddelbart efter det praktiske arbejde, kravet har været at der skulle vente eleverne noget skriftligt arbejde som følge af det praktiske arbejde de netop havde udført. Foruden de frem udsagn havde eleverne mulighed for at give kommentatere de måtte finde relevante. Undersøgelsen blev foretaget med Survey-Xact softwaren i de respektive klasser på skolen.  Efter at have indsamlet empiri er det på sin plads at undersøge om den indsamlede empiri er konsistent, dette gøres her med Cronbach's $\alpha$. 
\begin{table}[h!]
	\centering
	\caption{Her ses værdier for Cronbach's $\alpha$ som mål for at teste den interne konsistens i undersøgelsen. Data som disse kan findes i en lang række artikler, men her følger vi udlægningen af \citep[Tabel 1, s. 382]{Peterson1994} hvor der er en række forskellige fortolkninger, den her anvendte tager sit udgangspunkt her og er så tilpasset.}
	\label{tbl:alpha}
	\begin{tabular}{@{ } c c @{ }}
		\toprule[2.pt]
		Værdi af Cronbach's $\alpha$ & Intern konsistens\\
		\midrule
		$0.9\leq \alpha$ 		& Fremragende\\
		$0.8\leq \alpha < 0.9$ 	& God\\
		$0.7\leq \alpha < 0.8$	& Acceptabel\\
		$0.6\leq \alpha < 0.5$	& Tvivlsom\\
		$0.5\leq \alpha < 0.6$ 	& Dårlig\\
		$\alpha < 0.5$			& Uacceptabel\\
		\bottomrule[2pt]
	\end{tabular}
\end{table}
Cronbach's  $\alpha$ er beregnet på følgende vis: 
\begin{equation}\label{eq:alpha}
	\alpha = \left(\frac{k}{k-1}\right)\cdot \left(1-\sum_{i=1}^{k} \frac{\sigma^{2}_{i}}{\sigma^{2}{s}}\right)
\end{equation}
hvor $k$ er altallet af målinger i undersøgelsen, $\sigma^{2}_{i}$ er variansen af den i'te måling, mens $\sigma^{2}_{s}$ er variansen for hele undersøgelsen, jf \cite[s.382]{Peterson1994}. For de indsamlede data i spørgeskemaet er Cronbach's $\alpha$ beregnet til 0.91, hvilket baseret på skalaen i tabel \vref{tbl:alpha} betyder at der er en fremragende intern konsistens i undersøgelsen. Hvilket igen betyder at elevernes svar er konsistente gennem alle spørgsmålene. 


\subsection{Skriftligt arbejde}
\label{sub:3.2.b}
Den anden del af indsamlingen af data til at belyse problemstillingen her er i form af skriftlige arbejder fra eleverne. Her er tilfældigt udtrukket to grupper fra den klasse som er omdrejningspunktet for eleverne. Deres skriftlige arbejder er så indsamlet og analyseret med udgangspunkt i TAP metoden beskrevet af \citep{Erduran2004}. Hermed bliver målet af om eleverne faktisk udvilker deres skriftlige kompetencer altså vurderet i forhold til et sæt af objektive kriterier. 
For at belyse om IBSE og SWH faktisk har en effekt på elevernes skriftlige arbejde kigger vi samtid på skriftligt arbejde fra tilfældigt valgte elever fra to andre klasser på Viborg Katedralskole.

\section{Empiriske begrænsninger}
\label{sec:3.3}
Det er klart at der vil være nogle åbenlyse begrænsninger i forhold til den indsamlede empiri. Empirien er indsamlet fra en meget lille poppulation og det vil derfor ikke være muligt at konkluderer noget generelt på baggrund af den indsamlede empiri. Empirien kan dog hjælpe med at afdække tendenser i forhold til hvad der hjælper elevernes skriftlige arbejde på vej og hvad der ikke gør. Dette projekt skal med andre ord ses som et pilot projekt. 

En anden åbentlys begrænsning er at projektet gennemføres på en skole med en primær klasse som omdrejningspunkt. Man kunne forestille sig at der ville have været et andet udfald såfremt man havde gennemført projektet i en anden klasse på en anden skole, eller blot en anden klasse på den samme skole. Det vil jeg lade være op til fremtidig forskining inden for dette område at afgøre. 




%\section{Liste over empiriske muligheder}
%%\label{sec:3.3}
%
%Der er mange muligheder når vi skal se på udviklingen af elevernes faglige motivation samt udviklingen af elevernes skriftlige kompetence. Herunder følger en %liste som den ser ud pr. d.d. og denne er krydret med indspark fra \citet{Brinkmann2015}.
%\begin{enumerate}
%	 \item {\bfseries \color{red}Test} - jeg har tænkt på at man kunne gennemfører nogle tests som tester elevernes viden om faglige problemstillinger før %under og efter et forløb med fokus på skriftlighed og UBNU. En af udfordringerne her er at kunne sammenligne for at se på om de forbedre sig mere end andre, %og hvordan måler vi egentlig skriftlig kompetence og motivation for faget?
%	 \item {\bfseries \color{red}Aflevering} - Jeg har tænkt på om man kunne se på elevernes afleveringer sammenlignet med lignende klasser som ikke har et %særligt fokus på at kører deres forløb gennem UBNU og med særligt fokus på skriftlighed og formativ feedback. Igen så er her nogle faldgrupper i form af at vi %kan komme til at sammenligne æbler og pærer og ikke nødvendigvis være istand til at påvise det vi tror vi ser. 
%	 \item {\bfseries \color{red} Spørgeskema} - En tredje model er at bede eleverne udfylde et super kort spørgeskema fra tid til anden gennem forløbene for %at se på hvordan elevernes faglige motivation er koblet til det at lave praktisk arbejde i faget. Her vil der i  mindre grad være behov for at sammenligne på tværs %af klasser selvom det vil give en bedre baseline. Men der er igen mange faktorer som kan spille ind på elevernes oplevede motivation for et fag. Herunder både %ydre og indre motivation, jf \citet{Buhl2010}. Fordelen ved at have et spørgeskema er at man får samlet et statisktisk materiale som er forholdsvist til at have %med at gøre.
%	\item {\bfseries \color{red} Interview} - En anden mulighed kunne være at interviewe nogle elever om deres motivation for faget fysik og så efterfølgende se %på hvordan disse elevers oplevede motivation stemmer overens med lærernes oplevelse af elevernes motivation i forhold til faget og se på om motivation, %praktisk arbejde og skriftlighed faktisk går hånd-i-hånd.
%	\item {\bfseries \color{red} Selvevaluering} - En alternativ metode vi har snakket om var at lade eleverne selv evaluerere deres motivation for at %gennemfører det konkrete skriftlige produkt således at man i højere grad fik eleverne egne holdninger til den måde at arbejde på i spil. Dette kunne også gøre %på tværs af hold på en årgang og så på baggrund af elevernes svar udtage personer til et uddybende interview.
%	\item {\bfseries \color{red} Observationer} - Der tænkes at skulle foretages observationer i de klasser hvor jeg kommer til at hente forsøgspersoner til %projektet således at jeg kan se hvordan de bliver undervist i forhold til deres praktiske arbejde - altså om de gennemfører øvelser efter UBNU tanken mv.
%\end{enumerate}