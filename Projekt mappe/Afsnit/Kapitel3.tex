\chapter{Empiri}
\label{Ch:3}

\section{Undersøgelsesdesign}
\label{sec:und}
Da dette projekt beskæftiger sig med motivation for faget fysik samt for det skriftlige arbejde og udviklingen af den skriftlige kompetence hos en gruppe af elever. Kommer projektet naturligt til at anvende aktionsforskning. Der bliver altså gennemført en række intervensioner i forhold til praksis for undervisningen i en klasse. Samtidig holdes en anden klasse ude af eksperimentet med ændringen af praksis og kan derfor indgå som kontrol gruppe. Designet af undersøgelsen struktureret således at en klasse på en 1.g årgang er udvalgt som test klasse dette drejer sig om en klasse på Viborg Katedralskole med studieretningen matematik A og fysik A, herefter kaldet 1.y. Klassen 1.y blev undervist induktivt med stor grad af \ib{}, og med fokus på anvendelsen af 6F-modellen som vist på figur  \firef{6F}, gennem hele 1.g fysik forløbet, og der blev i klassen anvendt SWH som en del af udviklingen af klassens skriftlige kompetencer. I forhold til anvendelsen af SWH i undervisningen blev der lavet en lokal tilpasning til den model for SWH er præsenteret i afsnit \vref{sec:teo}. Denne lokale tilpasning har sigte på at give eleverne et dokument som de kan bruge som huskeliste i forbindelse med deres skriftlige arbejde. Her er faserne for forhandling som fremgår af \tbref{2.2} ændret en smugle for at tilpasse den til Viborg Katedralskole. Den største forskel er at flere af forhandlingsfaserne er lagt sammen så efterarbejdet reduceret en anelse, for en mere uddybende gennemgang af den lokale tilpasning, se appendix \vref{app:A} udgangspunktet for denne tilpasning er \citep{Greenbowe2005, Hand2004, Krogh2016}. For at undersøge elevernes motivation for det skriftlige arbejde blev eleverne bedt om at udfylde et spørgeskema lige efter de havde været i laboratoriet og forud for udførelsen af det skriftlige arbejde. På Viborg Katedralskole blev alle eleverne i 1.g udsat for dette spørgeskema efter et af de første moduler i laboratoriet, efter afslutningen på grundforløbet. 
For at kunne sammen ligne udviklingen af den skriftlige kompetence for eleverne i 1.y, blev der tilfældigt udvalgt skriftligt arbejde for to grupper i 1.y samt for to grupper af elever i 1.e som ikke er en science studieretning. Det er klart at det mest optimale havde været at have en klasse som var sammenlignlig med 1.y, dette var dog ikke muligt på Viborg Katedralskoles årgang 2018/19. Det skal dog her bemærkes at sammenligningen mellem to forskellige studieretninger ikke nødvendigvis er valid og derfor kan give problemer i forhold til motivations undersøgelsen.

\section{Indsamlingsmetoder}
\label{sec:ind}
Til indsamling af den anvendte empiri er der anvendt hhv. spørgeskema og afleveringer fra eleverne. Herunder gennemgåes præmisserne for indsamlingen af spørgeskemaerne.

\subsection*{Spørgeskemaet}
Eleverens selvevaluering af deres eget motivations niveau, har mundet ud i en udformning af et spørgeskema med fem udsagn som eleverne skal vurdere på en syv trinsskala. Hvor syv er meget enig og et er meget uenig. De fem udsagn som eleverne skal vurdere er følgende:
\begin{enumerate}
	\item Jeg har let ved at gennemskue hvad jeg skal i laboratoriet.\vspace{-15pt}
 	\item Jeg har et øget fagligt udbytte af de åbne problemstillinger.\vspace{-15pt}
	\item Jeg har en bedre forståelse af den teori der arbejdes med som følge af laboratorie arbejdet.\vspace{-15pt}
	\item Jeg føler at skriveprocessen er nemmere når jeg selv har designet forsøget.\vspace{-15pt}
	\item Jeg føler at det praktiske arbejde i laboratoriet, øger min faglige motivation.
\end{enumerate}
Spørgeskemaet gennemføres i en eksperimentel lektion umiddelbart efter det praktiske arbejde, kravet har været at der skulle vente eleverne noget skriftligt arbejde som følge af det praktiske arbejde de netop havde udført. Foruden de frem udsagn havde eleverne mulighed for at give kommentatere de måtte finde relevante. Undersøgelsen blev foretaget med SurveyXact softwaren i de respektive klasser på skolen.

\subsection*{Afleveringer}
Den anden type af empiri der er anvendt til dette projekt er elev arbejde. Her har underviserne tilfældigt udvalgt to grupper pr klasse forud for første aflevering og disse gruppers afleveringer er således blevet gennemlæst for at finde tegn på udvikling af den skriftlige kompetence i både 1.e og 1.y. Den anden aflevering i dette datasæt er udvalgt således at det følger den person i gruppen som har afleveret den første opgave i Lectio studieadministrations systemet såfremt eleverne har skiftet gruppe undervejs i forløbet. 

\subsection*{Samtalen}
Som den sidste del af den foreliggende empiri er der gennemført en klassedialog med klassen 1.y om motivation og særligt fokus på motivation for skriftligt arbejde - endvidere har klassen i denne forbindelse givet deres bud på hvordan man ville kunne øge deres motivation for det skriftlige arbejde. Samtalen var i udgangspunktet styret af klassens undeviser, som undervejs noterede de  vigtigste pointer på tavlen således at de blev bevaret til opgaven.

\section{Diskussion af undersøgelsesdesignet}
\label{sec:des}
I forbindelse med dette projekt har der været anvendt metoden aktionsforskning, hvor der er blevet ændret på ting i forhold til undervisningen for derefter at se hvilken effekt det har på eleverne. Grundlaget for dette valg har været af praktiske hensyn, på den måde er det muligt at afprøve forskellige tiltag på et mindre ensemble for derefter at undersøge effekterne af de ændringer der er foretaget blandt målgruppen. Ved evalueringen af de forskellige tiltags effekter på eleverne ville det give mening at udføre pretests og posttests for derigennem at undersøge elevernes faglige udbytte. Dette er bevidst valgt fra da fokus her har været på at undersøge elevernes motivation for faget samt indflydelsen på den skriftlige kompetence. Her ville før og efter tests ikke nødvendigvis kunne give et retvisende billede. Hvorfor spørgeskemaet som en form for selvevaluering virker som en mere fornuftig vej at gå.  En anden begrænsende faktor her har været tidsfaktoren. Skulle man have foretaget undersøgelserne på et større ensemble så ville det have krævet langt flere undervisere som var indvolveret i forskningen. Dette har ikke været muligt, hvorfor valget er faldet på at studerer effekterne i en enkelt klasse med 30 elever. Endvidere ville det også være sværer at sikre at alle underviserne så rent faktisk udførte undervisningen i henhold til den struktur som de havde fået udstukket. 
