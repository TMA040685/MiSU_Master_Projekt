\chapter{Teori}
\label{Ch:2}

I dette kapitel ser vi på den bagvedliggende teori på området og hvorfor det er interessant at se på dels hvilke studier der er foretaget på området, men i ligeså høj grad hvilke studier der mangler inden for fx det danske skole system. Kapitlet er opbygget således at der først er en indføring i den litteratursøgning der er gennemført i forbindelse med dette speciale, samt en beskrivelse af afgrænsningen for litteratur udvælgelsen. Dette findes i afsnit \vref{sec:2.1} og efterfølges af en syntese hvor der uddrages temaer fra de artikler der er fundet gennem litteratur søgningen, afsnit \vref{sec:2.2}.

\section{Litteratursøgning}
\label{sec:2.1}


\begin{wrapfigure}{o}{0.5\textwidth}
	\centering
	\smartdiagram[descriptive diagram] {
		{Trin 1, 335\,000 hits via Søgning på Google Scholar},
		{Trin 2, 21\,551 hits via søgning på library.au.dk},
		{Trin 3, 15 hits efter screening af abstracts},
		{Trin 4, 10 hits },
		{Trin 5, 8 hits},
	}
	\caption{Litteratursøgnings processen}
	\label{fig:21a}
\end{wrapfigure}

Processen med at udvælge primær litteratur til dette speciale er forløbet i henhold til følgende principper, som ligeledes er illustret på  figur\vref{fig:21a}. Jeg har valgt at dele denne process ind i 5 trin som er gennem løbet på følgende vis og gennem følgende kriterier. Først gennemførtes en søgning på Google scholar med følgende udsagn ``\emph{practical work AND science writing heuristic}''. Det er klart at der er behov for en grovere sortering med et udgangspunkt på 335\,000 hits. 
I trin to lavede jeg en sørning på \url{http://library.au.dk} med den samme søge tekst og med et ekstra kriterium nemlig at kilderne skulle være skrevet fra og med 1999. For at få nyere forskning tilgængelig dette reducerede antallet af hits til 21\,551. Det blev derfor besluttet udelukkende at lede efter artikler som indeholdt enten ordet fysik eller kemi da disse fag er meget nært beslægtede og der vil være en fællesmængde af overførbar viden mellem de to fag i forhold til de didaktiske udfordringer.


\section{Litteraturreview}
\label{sec:2.2}

\lipsum*