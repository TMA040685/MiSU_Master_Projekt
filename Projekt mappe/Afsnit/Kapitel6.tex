\chapter{Konklusion}
\label{Ch:6}

Med udgangspunk i analysen af den insdamlede empiri, kan det konkluderes at det på nuværende tidspunkt ikke er muligt at sige noget om effekten af udervisningen efter \ib{}-principperne i samspil med SWH på elevernes motivation for det skriftlige arbejde. Dette lader ikke til at have ændret sig. Dette skyldes problemer med det empiriske grundlag hvor det ikke var muligt at hente test klassens svar ud af spørgeskemaet uafhængigt af de øvrige besvarelser, grundet anonymitet.

Gennem samtaler med eleverne kom det frem at de i højere grad motiveres af mundtlig feedback også på skriftligt arbejde. Dette skyldes at eleverne har lettere ved at fortolke den mundtlige feedback, i forhold til den skriftlige feedback. Her bør det dog problematiseres at eleverne får sværer ved at fastholde deres fokuspunkter med mindre disse overdrages skriftligt. Fokuspunkterne er meget vigtige i forhold til den fortsatte udvikling, og for at det bliver tydeligt for eleverne hvor de er på vej hen jf. \citep{Hattie2015}. Eleverne peger også på at mere tid i den eksperimentelle fase vil gøre det skriftlige arbejde lettere og dermed øge deres motivation for det. 
Eleverne giver altså idéer til hvordan man ville kunne øge den faglige motivation blandt elever i forhold til skriftlig fysik, her med fokus på skriftligt arbejde i form af rapporter og journaler. 

Betragtes nu svarene fra 1.g eleverne givet i spørgeskemaet, er det åbentlyst at den største faglige udfordring for fysikfaget er den skriftlige motivation. For eleverne svarer jf. figur \firef{4.1.a} at de generelt har en god faglig motivation og at de let kan gennemskue hvad de skal i laboratoriet. Det eneste punkt hvor eleverne svarer forholdsvist lavt er på spørgsmålet om den skriftlige motivation, dog er svarene stadig inden for den statistiske usikkkerhed på målingen. Eleverne er finder det fagligt motivativerende at foretage praktisk arbejde hvilket også afspejles i det indtryk de fleste fysik undervisere har, jf. \citep{Krogh2016}. Dette står endvidere i opposition til de slutninger som drages af \citep{Hodson2008}.

Fra analysen af fremstår det tydeligt at elev produkterne i 1.y er der en højere grad af analytisk og reflektiv tilgang til det praktiske arbejde. Der er heller ikke på samme måde berøringsangst med eksperimenter som ikke giver det forventede udfald. I klasser som ikke er undervist efter \ib{} og SWH, oplever man at eleverne frustreres over data som ikke passer med den teori de skulle eftervise, eller som slet ikke reflektere over at der er en mulig uoverensstemmelse mellem teori og data. Dette er bestemt ikke tilfældet med test klassen i denne kontekst. Det kan derfor sluttes at elevernes skriftlige kompetencer er øget betydeligt i forhold til kontrol klassen 1.e. Denne slutning ligger fint i tråd med andre studier fx \citep{Atasoy2013, Erkol2010, Burke2005}.

Sluttelig kan det konkluderes at samspillet mellem \ib{} og SWH giver mening i forhold til at udvikle elevernes kritiske rationelle sans i forhold til eksperimenter og analysen af de indsamlede eksperimentelle data. I forhold til elevernes faglige motivation kan man ud fra betragtininger af eleverne i undervisingen uddrage at \ib{} øger motivationen for det faglige arbejde og for undren over fænomener inden for fysikkens dogmæne, som beskrevet hos \citet{Dolin2014}. Det står også klart at SWH metoden understøtter netop den reflektive process i fysik faget. Udfordringerne ved at arbejde på denne metode er at det er meget tidskrævende, og i et presset pensum vil det ofstest være et sted hvor man som underviser vælger fra. 

Dette pilotprojekt har nogle interessante perspektiver men der er et klart behov for yderligere undersøgelser af de effekter som dette projekt indikerer der kunne være. Særligt i forhold til skriftlig motivationen af eleverne og for at opnå yderligere dokumentation for løftet af elevernes skriftlige kompetencer. 


\section{Perspektivering}
\label{sec:per}

En af de store udfordringer i brugen af \ib{} og SWH er at dette tager meget længere tid end at gennemfører øvelser på denne måde. Dette kan udelukkende få nogle undervisere til at vælge denne undervisningsform fra. Derfor er det tvingende nødvendigt at der bliver lavet flere studier i en dansk kontekst som ville kunne underbygge tesen om at eleverne bliver fagligt dygtigere ved at modtage undervisning på denne måde. Resultaterne af denne undersøgelse tyder på at eleverne ved at modtage undervisning på denne måde ikke kun øger deres fag faglige kompetencer men også deres almene laboratorietekniske kompetencer, hvilket også er noget man bør træne i dagligdagen, men som fra tid til anden forsvinder i opfyldelse af hvilke øvelser der skal gennemføres.

Samtidig med at fysik faget i disse år gennemgår en udvikling hvor flere og flere undervisere vender de gammeldags kogebogsvejledninger ryggen og leder efter alternativer,  så kunne et velformuleret alternativ i form af \ib{} og SWH være en metode som man let ville kunne adoptere og implementere i sin daglige undervisning. 

Det vil altså være nødvendigt med et yderligere studium af samspillet mellem \ib{} og SWH. Her bør man overveje et større ensemble af elever og ligeledes et bredere spektrum i forhold til studieretninger, ligeledes bør man også have et bredere udsnit af undervisere inden for fysik faget som kan indgå i projektet. Hertil kommer en kontrolgruppe som gennemfører projekt perioden men med klassiske kogebogsvejledninger.



