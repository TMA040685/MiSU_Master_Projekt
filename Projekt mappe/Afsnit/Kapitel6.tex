\chapter{Konklusion}
\label{Ch:6}

Med udgangspunkt i den indsamlede empiri og den gennemførte analyse af denne empiri, som danner grundlaget for dette pilot projekt, kan det konkluderes at det ikke er muligt i disse data at se nogle effekter på elevernes motivation for skriftligt arbejde. Hvilket skyldes problemer med den foreliggende empiri. Eleverne udtrykker når de bliver adspurgt at de har en meget højere lyst til at lære i forhold til det mundtlige arbejde frem for det skriftlige arbejde. Eleverens begrundelse herfor er at de føler at de har lettere ved at fortolke og afkode den feedback de får mundtligt, frem for når de modtager den på skrift. Udfordringen kan her være fastholdelse af feedbacken i forhold til at gøre den til feedforward, således at det bliver tydeligt hvad eleverne skal tage med videre, jf. slutningerne præsenteret af \citet{Hattie2015}. Eleverne giver altså idéer til hvordan man ville kunne øge den faglige motivation blandt elever i forhold til skriftlig fysik, her med fokus på skriftligt arbejde i form af rapporter og journaler. Generelt set kan det uddrages af figur \firef{4.1.a} er at eleverne generelt finder faglig motivation i det praktiske arbejde hvilket også afspejles i det indtryk de fleste fysik undervisere i det almene gymnasium påpeger, jf. \citep{Krogh2016}.

Fra analysen af fremstår det tydeligt at elev produkterne i 1.y er der en højere grad af analytisk og reflektiv tilgang til det praktiske arbejde. Der er heller ikke på samme måde berøringsangst med eksperimenter som ikke giver det forventede udfald. I klasser som ikke er undervist efter \ib{} og SWH, oplever man at eleverne frustreres over data som ikke passer med den teori de skulle eftervise, eller som slet ikke reflektere over at der er en mulig uoverensstemmelse mellem teori og data. Dette er bestemt ikke tilfældet med test klassen i denne kontekst. Det kan derfor sluttes at elevernes skriftlige kompetencer er øget betydeligt i forhold til kontrol klassen 1.e.

Sluttelig kan det konkluderes at samspillet mellem \ib{} og SWH giver mening i forhold til at udvikle elevernes kritiske rationelle sans i forhold til eksperimenter og analysen af de indsamlede eksperimentelle data. I forhold til elevernes faglige motivation kan man ud fra betragtininger af eleverne i undervisingen uddrage at \ib{} øger motivationen for det faglige arbejde og for undren over fænomener inden for fysikkens dogmæne, som beskrevet hos \citet{Dolin2014}. Det står også klart at SWH metoden understøtter netop den reflektive process i fysik faget. Udfordringerne ved at arbejde på denne metode er at det er meget tidskrævende, og i et presset pensum vil det ofstest være et sted hvor man som underviser vælger fra. \bigskip

Denne pilot undersøgelse har nogle interessante perspektiver men der er et klart behov for yderligere undersøgelser af de effekter som dette projekt indikerer der kunne være. 


\section{Perspektivering}
\label{sec:per}

En af de store udfordringer i brugen af \ib{} og SWH er at dette tager meget længere tid end at gennemfører øvelser på denne måde. Hvis blot man kan dokumentere at elevernes kompetencer udvikles mens de arbejder. Med udgangspunkt i dette pilot studie er der indicier for at elevernes skriftlige kompetencer forbedres betydeligt i forhold til den klassiske kogebogsøvelse. Der er noget der tyder på at eleverne bliver dygtigere til at gennemfører og analysere praktisk arbejde og de resultater som kommer ud af den empiri som eleverne indsamler gennem det praktiske arbejde.\bigskip


Det vil altså være nødvendigt med et yderligere studium af samspillet mellem \ib{} og SWH. Her bør man overveje en større poppulation af elever og ligeledes et bredere spektrum i forhold til studieretninger, ligeledes bør man også have et bredere udsnit af undervisere inden for fysik faget som kan indgå i projektet. Hertil kommer en kontrolgruppe som gennemfører projekt perioden men med klassiske kogebogsvejledninger.\bigskip



