\chapter{Diskussion og opsummering}
\label{Ch:5}

\section{Hvordan påvirkes 1.g elever af \ib{} og SWH?}
\label{sec:ibseswh}
Hvad kan der så uddrages af denne undersøgelse af elevernes motivation for det skriftlige arbejde og udviklingen af deres skriftlige kompetencer ved brug af \ib{} og SWH? Det bliver svært at uddrage noget generelt på baggrund af dette relativt lille pilot studie på en relativ beskeden poppulation af elever. Dog kan det uddrages at eleverne ikke er helt trygge ved de frie rammer som kommer med \ib{}-metoden. De efterlyser mere styrring, til trods for dette fremgår det med alt tydelighed af deres afleveringer at de har et højt udbytte af den mere frie tilgang som \ib{} giver dem. Samspillet med SWH giver en rigtig velstruktureret argumentation som indikerer et højt indlæringsniveau for eleverne. 

\begin{figure}[h!]
	\centering
	\usetikzlibrary{decorations.text}
	\begin{tikzpicture}
		%\draw[very thin, red!25, step=1 mm, anchor= south east] (0,0) grid (12,12);
		%\draw[thick, red!50, step=5 mm, anchor= south east] (0,0) grid (12,12);
		%\draw[very thick, red!75, step=10 mm, anchor= south east] (0,0) grid (12,12);
		
		\filldraw[very thick, draw=black, fill=red!50] (6,6) circle (5cm);
		\filldraw[very thick, draw=black, fill=white] (6,6) circle (4cm);
		\filldraw[very thick, draw=black, fill=blue!50] (6,6) circle (2.5cm);
		
		\draw[very thick] (0.5,6) -- (4.5,6) (7.5,6) -- (11.5,6) (6,0.5) -- (6, 4.5) (6,10.75) -- (6,11.5) (6,7.5) -- (6,8.75);
		
		\node at (6,6) {Tryghedszonen};
		\path [decorate, decoration= {text along path, text={Zonen for nærmeste læring}}]
		(3.75,8) arc (140:0:3cm);
		\path [decorate, decoration= {text along path, text ={Utryghedszonen}}]
		(4.75,10) arc (110:0:3.95cm);
	\end{tikzpicture}
	\caption[Vygotskys udviklingszoner]{Grafisk model af Vygotskys udviklingszoner, her inddelt i tryghedszonen hvor man kan løse de opgaver man stilles over for, zonen for nærmeste læring hvor man skal have hjælp til at løse opgaverne men hvor man gennem mestring vil blive dygtigere og sluttelig utryghedszonen hvor man mister håbet. }
	\label{fig:vyg}
\end{figure}

Eleverne oplever stor frustration over de mere frie opgaver som stilles, dette skyldes at eleverne mener at der er noget som er de gode eksperimenter og noget som er dårlige eksperimenter. Eleverne er sikker på at der er et endeligt facit med opgaven. Skal man være tro mod principperne i \ib{} skal eleverne selv drive eksperimentet dermed vil der aldrig blive noget egentligt facit på eksperimenterne. Eleverne frustreres ligeledes over hvad de betegner som dårlige data, med dette mener de data hvor de har svært ved at uddrage noget. Dette er imidlertid en pædagogisk pointe at alle data kan fortælle noget om den virkelighed der undersøges og dermed vil det pege tilbage i det oprindelige undersøgelsesspørgsmål, endvidere vil disse data med tilhørende undersøgelsesspørgsmål fortælle noget om elevernes nuværende kognitive faglige niveau.  Elevernes frustrationer her skyldes at de tvinges ud i zonen for nærmeste læring, se figur \firef{vyg}, og derved får muligheden for at udvikle sig, men det kommer med en pris. Prisen for denne udvikling er at man er nødt til at arbejde og at man gør en indstats for at udvikle sig. Eleverne vil helst som de fleste andre mennesker helst befinde sig i deres trykhedszone hvor de har helt styr på alle ting og de udfordringer de stilles overfor. Underviser man eleverne således at de ikke flyttes ud af denne zone vil man få elever som ikke udvikler sig og dermed ikke tilegner sig nye færdigheder. 

Ser man bort fra elevernes frustrationer over at de har svært ved at arbejde på denne nye og anderledes måde at tilgå viden i fysikfaget når man ikke starter med teorien men derimod med fænomenet. Så fremgår det af de undersøgte afleveringer at elevernes evner inden for faget er rykket betydeligt. Eleverne har med andre ord rykket grænsen for deres trykhedszone, således at denne er blevet udvidet. Med denne udvidelse af trykhedszonen vil også zonen for nærmeste læring også udvide sig. Denne udvikling er eleverne ikke nødvendigvis bevidste om men det står klart for underviserne som har med eleverne at gøre til daglig.

\section{Faglig motivation}
\label{sec:faglig}
Klassen 1.y har en høj grad af faglig motivation. Eleverne i klassen har aktivt valgt fysik og matematik som studieretningsfag. Det er dermed svært at finde elever som er mere motiverede for faget end netop disse elever. Udfordringen her er altså hvorvidt det er muligt at dokumenterer ændringer i den faglige motivation som følge af undervisningen efter \ib-tanken med særligt fokus på SWH. Problemet her er at eleverne i klassen 1.y's svar på spørgeskemet ikke er tilgængelige som egne svar uden de øvrige 1.g elevers svar. Derved drukner eventuelle informationer om elevernes faglige motivation med fokus på det skriftlige arbejde, som følge af praktisk arbejde, i baggrunden af de øvrige svar fra 1.g'erne på Viborg Katedralskole.
Af samtalen med klassen 1.y fremgik det at de følte sig højere motiveret ved mundtligt arbejde grundet den direkte feedback og at de oplever at de lettere kan fortolke den feedback de får fra underviser og klassekammerater. Men de har mange bud på hvordan man ville kunne øge deres faglige motivation i forhold til det skriftlige arbejde. Her vil idéer som at skrive rapporterne i mindre bidder såvel som at have skrive moduler umiddelbart efter et modul med praktisk arbejde. Ligeledes vil det at dele opgaveskrivningen op i mindre bidder også give mulighed for at give eleverne noget mere mundtlig feedback på det skriftlige arbejde. 

\section{Skriftlige kompetencer}
\label{sec:skr}
Tilbage står nu hvordan det så er gået med de skriftlige kompetencer hos eleverne i 1.y kontra de skriftlige kompetencer i klassen 1.e. Her er det tydeligt at når man ikke har en helt tydelig veldefineret opgave i forhold til nøjagtig hvilke bergninger man skal foretage for at nå i mål med det skriftlige arbejde man netop er i gang med. Det er fra analysen klart at eleverne i 1.y har udviklet sig i meget højere grad i forhold til det skriftlige kompetence sæt der er beskrevet med TAP-modellen af \citet{Erduran2004}. Foruden en mere fornuftig argumentation i det skriftlige arbejde ses også et mere velreflekteret skriftligt produkt hvor eleverne i 1.y afsøger mange flere mulige forklaringer på deres problemstillinger. Endvidere lader det til at eleverne i højere grad formår at forholde sig konstruktive til eksperimenter som ikke har det forventede udfald. 