\chapter{Diskussion og opsummering}
\label{Ch:5}

\section{Hvordan påvirkes 1.g elever af \ib{} og SWH?}
\label{sec:ibseswh}
Hvad kan der så uddrages af denne undersøgelse af elevernes motivation for det skriftlige arbejde og udviklingen af deres skriftlige kompetencer ved brug af \ib{} og SWH? Det bliver svært at uddrage noget generelt på baggrund af dette relativt lille pilot studie på en relativ beskeden poppulation af elever. Dog kan det uddrages at eleverne ikke er helt trygge ved de frie rammer som kommer med \ib{}-metoden. De efterlyser mere styrring, til trods for dette fremgår det med alt tydelighed af deres afleveringer at de har et højt udbytte af den mere frie tilgang som \ib{} giver dem. Samspillet med SWH giver en rigtig velstruktureret argumentation som indikerer et højt indlæringsniveau for eleverne. 
Eleverne oplever stor frustration over de mere frie opgaver som stilles, dette skyldes at eleverne mener at der er noget som er de gode eksperimenter og noget som er dårlige eksperimenter. Eleverne er sikker på at der er et endeligt facit med opgaven. Skal man være tro mod principperne i \ib{} skal eleverne selv drive eksperimentet dermed vil der aldrig blive noget egentligt facit på eksperimenterne. Eleverne frustreres ligeledes over hvad de betegner som dårlige data, med dette mener de data hvor de har svært ved at uddrage noget. Dette er imidlertid en pædagogisk pointe at alle data kan fortælle noget om den virkelighed der undersøges og dermed vil det pege tilbage i det oprindelige undersøgelsesspørgsmål, endvidere vil disse data med tilhørende undersøgelsesspørgsmål fortælle noget om elevernes nuværende kognitive faglige niveau. 

\section{Faglig motivation}
\label{sec:faglig}


\section{Skriftlige kompetencer}
\label{sec:skr}