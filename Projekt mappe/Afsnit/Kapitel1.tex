\chapter{Introduktion}
\label{Ch:1}

Jeg ønsker med denne opgave at arbejde med skriftligheden i fysik faget. Den praktiske dimension af de naturvidenskabelige fag har tidligere været et adelsmærke for hvordan vi tænker faget motiverende for eleverne. Nogle studier peger på at det ikke nødvendigvis forholder sig sådan længere. Det er interessant at undersøge om man gennem en indsats i forhold til det pratiske arbejde og det skriftlige arbejde kan øge elevernes faglige motivation i forhold til faget fysik. 
\section{Motivation}
\label{sec:1.1}
Gennem de seneste syv år har jeg arbejdet med undervisning i fysik i det almene gymnasium, herefter STX, gennem de syv år er det blevet klart for mig at det vi som undervisere tænkger er sjovt og motiverende for vores elever ikke nødvendigvis er det. Dette underbygges bl.a. af \citep{Hodson2008} hvor i det beskrives hvorledes eleverne faktisk udelukkende finder ekspeimenterne som en adspredelse fra den ellers kedelige teoriundervisning, men de bidrager ikke i positiv forstand til elevernes motivation for faget.  \citet[]{Hodson2008} går så langt som til at antyde at det kan være direkte kontra produktivt at gennemfører øvelser med eleverne. Så hvad kan vi gøre for at øge elevernes motivation for naturfagene og i særdeleshed for fysik? Her peger flere undersøgelser på at Undersøgelsesbaseret naturfags undervisning (UBNU) kan være en vej til øget motivation \citep{Krogh2016, Dolin2014} for eleverne. 

{\color{red} TO DO: Færdiggøre argumentationen omhandlende motivation - herunder bør det udpeges at en lignende undersøgelse ikke er foretaget i Danmark - med forfatterens vidende.}


\section{Problemformulering}
\label{sec:1.2}
Jeg har valgt at fokusere på elevernes faglige motivation i forbindelse med  \emph{inquriy based science education} arbejdet endvidere har jeg valgt at understøtte IBSE tanken ved brugen af \emph{science writing heuristic} i det skriftlige arbejde. Min problem formulering lyder derfor som følger.
\begin{quote}
	{\bfseries Hvordan påvirkes 1.g elever af IBSE og SWH med særligt fokus på deres faglige motivation samt deres skriftlige kompetence?}
\end{quote}
 
%\lipsum*