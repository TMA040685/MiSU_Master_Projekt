\chapter{Introduktion}
\label{Ch:1}

Denne opgave beskæftiger sig med elevernes skriftlige arbejde i fysik faget, herunder undersøges elevernes motivation i forhold til både det praktiske arbejde og til det skriftlige arbejde med faget. Den praktiske dimension af de naturvidenskabelige fag har tidligere været et adelsmærke for hvordan fagene tænkes at kunne motivere for eleverne. Studier som fx \citep{Hodson2008} peger på at man måske overfortolker betydningen af det praktiske arbejde som motivator for eleverne. Det er interessant at undersøge sammenhængen mellem det praktiske arbejde og elevernes motivation for faget og for skriftligheden, i en dansk kontekst da \citep{Hodson2008} ikke inddrager danske data. Alle de studier som forelægger på nuværende tidspunkt omhandler enten elever i grundskolen eller elever på universitet, og ingen af studierne er udført i en dansk kontekst. Derfor er det interessant at undersøge hvorledes man kan påvirke elevernes læring ved at ændre på den måde hvorpå der undervises i fysikfaget og den tilgang der anvendes til det skriftlige arbejde i faget. 

\section{Motivation}
\label{sec:1.1}
Når man som underviser planlægger sin undervisning inden for naturfagene sørger man i stort omfang for at indlægge eksperimentelle undersøgelser for at skabe adspredelse og motivation for faget blandt eleverne. Grunden til at mange undervisere tænker at det praktiske arbejde kan have motiverende effekt på eleverne, er sandsylingvis at underviserne selv oplevede det praktiske arbejde i fagene som motiverende. Desværre er der noget som tyder på at eleverne ikke nødvendigvis deler undervisernes opfattelse af den motiverende effekt ved praktisk arbejde i fagene jf. \citep[s. 65 - 69]{Krogh2016}. 
Dette underbygges bl.a. af \citep{Hodson2008} hvor i det beskrives hvorledes eleverne faktisk udelukkende finder ekspeimenterne som en adspredelse fra den ellers kedelige teoriundervisning, men de bidrager ikke i positiv forstand til elevernes motivation for faget.  \citet[]{Hodson2008} går så langt som til at antyde at det kan være direkte kontra produktivt at gennemfører øvelser med eleverne. 
Undersøgelsen som er udført af \citet{Hodson2008} er foretaget i en ikke dansk kontekst og derfor kunne det være interessant at undersøge om det er muligt at øge elevernes motivation gennem et øget fokus på undersøgelser og skriftlighed i fysik faget. For at øge elev motivationen tænkes det at arbejde med Inquiry Based Science Education, herefter blot IBSE\def\ib{IBSE}. \ib{} tilskrives en motiverende effekt for eleverne af \citep{Krogh2016, Dolin2014}. Da fokus er på at øge motivationen gennem et fokus på skriftlighed, anvendes den skriftlige metode kaldet Science Writing Heuristic, herefter SWH\def\sw{SWH}, son den beskrives af \citep{Keys1999, Krogh2016}. Derfor er problemformuleringen for dette projekt følgende.

\section{Problemformulering}
\label{sec:1.2}
Projektets problemformulering lyder som følger.
\begin{quote}
	{\bfseries Hvordan påvirkes 1.g elever af IBSE og SWH med særligt fokus på deres faglige motivation samt deres skriftlige kompetence?}
\end{quote}

\section{Projektets struktur}
\label{sec:1.3}

Dette projekt har karakter af et pilot projekt med et beskedent ensemble til at teste på. Derfor vil de konklusioner som kan uddrages af nærværende undersøgelse blot give indikationer på mulige sammenhænge som efterfølgende kan studeres i større detalje. Projektet er opbygget således at kapitel \vref{Ch:2} omhandler baggrunden for undersøgelsen herunder opgavens læringsteoretiske ståsted og den bagvedliggende teori, samt litteratur syntese og review af væsentlige dele af den anvendte litteratur. I kapitel \vref{Ch:3} beskrives det anvendte undersøgelses design samt de metoder der har været anvendt ved indsamlingen kapitlet slutter med en diskusion af muligheder og udfordringer ved dette empiriske design. I kapitel \vref{Ch:4} analyseres og fortolkes den indsamlede empiri og der uddrages resultater med udgangspunkt i analysen af empirien. Kapitel \vref{Ch:5} samler op på den analyse og de resultater der er uddraget i kapitel 4 og diskutere mulige forklaringer med afsæt i det læringsteoretiske ståsted som opgaven udspringer af som beskrevet i kapiten 2. Sluttelig drages der en samlet konklusion i kapitel \vref{Ch:6}, hvor der også er et bud på en perspektivering til den daglige undervisning i fysikfaget.
 