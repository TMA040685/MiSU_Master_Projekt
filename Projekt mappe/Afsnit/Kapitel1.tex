\chapter{Introduktion}
\label{Ch:1}

I denne opgave skal vi se på den skriftlige dimension af fysik undervisningen i det almene gymnasium. Gennem de senere år har faget gennemgået en didaktisk reformation og derfor er der et behov for at hjælpe eleverne på vej til en bedre forståelse da der i befolkningen begynder at være en holdning af at naturfag er svært og at dette er okay. 

\section{Motivation}
\label{sec:1.1}
Gennem de seneste syv år har jeg arbejdet med undervisning i fysik i det almene gymnasium, herefter STX, gennem de syv år er det blevet klart for mig at det vi som undervisere tænkger er sjovt og motiverende for vores elever ikke nødvendigvis er det. Dette underbygges bl.a. af \citep{Hodson2008} hvor i det beskrives hvorledes eleverne faktisk udelukkende finder ekspeimenterne som en adspredelse fra den ellers kedelige teoriundervisning, men de bidrager ikke i positiv forstand til elevernes motivation for faget.  \citet[]{Hodson2008} går så langt som til at antyde at det kan være direkte kontra produktivt at gennemfører øvelser med eleverne. Så hvad kan vi gøre for at øge elevernes motivation for naturfagene og i særdeleshed for fysik? Her peger flere undersøgelser på at Undersøgelsesbaseret naturfags undervisning (UBNU) kan være en vej til øget motivation \citep{Krogh2016, Dolin2014} for eleverne.


\section{Problemformulering}
\label{sec:1.2}

\lipsum*